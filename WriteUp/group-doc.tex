\documentclass{article}
\usepackage{multicol}
\usepackage{lipsum}
\newcommand{\frontmatter}{\pagenumbering{roman}\setcounter{secnumdepth}{0}}
\newcommand{\body}{\clearpage\pagenumbering{arabic}\setcounter{secnumdepth}{2}}

\title{%
Chop Chop \\
  \large Group Report}
\author{
  Connor, Wood\\
  \texttt{cw668}
  \and
  Will, Haysom\\
  \texttt{wah20}
  \and
  Dean, Gooding\\
  \texttt{Dmag2}
  \and
  Beth, Hill\\
  \texttt{Berh2}
  \and
  Jack, Fermer\\
  \texttt{Jdf50}
}

\begin{document}

  \maketitle

  \pagebreak

  \pagenumbering{roman}

  \begin{abstract}
    \addcontentsline{toc}{section}{Abstract}
      This report surrounds our group software development project, a smart cooking assistant called \emph{Chop Chop}. 
    
      In this report we will showcase our group project. We will be evaluating not only the software we made but also how we made it, looking into our design, development, and ways of working. In addition to this, we will 
  \end{abstract}

  \pagebreak

  \tableofcontents

  \pagebreak

  \pagenumbering{arabic}

  \begin{multicols}{2}

    \section{Introduction}
Chop Chop makes cooking more accessible by offering hands-free, smart recipes. Our smart recipes allow users to get on with cooking without having to worry about looking through the steps of a recipe, scrolling a screen with messy hands, or starting timers. It achieves this by using artificial intelligence, computer vision, and a full-stack application.
    \lipsum[3-7]

    \section{Background}
    \lipsum[31-35]

    \section{Aims}
    \lipsum[7-11]

    \section{Ways of Working}
    \subsection{Source Control}
    One of the key components to Chop-Chops’ success was our well-set-up and organised GitLab repository. 

    Initially, we decided to create a Gantt chart to record the project timeline and delegate deliverables, which was broken down into two-week sprints. Once all members approved this, it was sent to our supervisor who provided insight into improvements to ensure the project worked effectively. An example of this would be adding our module assessments to the chart to accurately manage and consider our university commitments and workload. 

    A scrum-style agile approach was adopted to our project management strategy. We broke down our Gantt charts’ deliverables into “Issues” on Gitlab. Issues were put on our virtual board, which had four categories, Open, In Development, Review/ Testing, and Closed. These issues were assigned and then moved across given their completion state. 

    From our team’s year-in-industry experience, we adopted a review before merging convention. This rule ensures that tasks were overlooked by other members before being added to the main branch from their working one. This had two main advantages. One was a hugely reduced chance of buggy code getting into the main branch. Two: you gained knowledge of how the rest of the project worked outside of your assigned issues. To enforce this, we locked the main branch from being merged and required at least one approval before the merge.

    We aimed to ensure the main branch was kept as clean as possible; we were mindful of what should and should not be on the repo. This included adding rules to the git ignore file when needed, pruning unnecessary files, and ensuring consistency in naming conventions for files and folders.

    \subsection{Meetings}
    \lipsum[13-15]

    \section{Design}
    \subsection{Frontend}
    \lipsum[15-17]
    \subsection{Architecture}
    \lipsum[17-19]

    \section{Development}
    \subsection{Backend}
    \lipsum[19-21]
    \subsection{Frontend}
    \lipsum[39-41]
    \subsection{Database}
    \lipsum[41-43]
    \subsection{Data Collection \& Training}
    \lipsum[21-23]

    \section{Testing}
    \lipsum[23-27]

    \section{Results}
    \lipsum[43-45]

    \section{Future Work}
    \lipsum[35-39]

    \section{Conclusion}
    \lipsum[27-31]

    \pagebreak

    
  \end{multicols}
  
  \addcontentsline{toc}{section}{References}
  \section*{References}
  References

\end{document}